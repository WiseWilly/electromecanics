\documentclass[a4paper,12pt,notitlepage]{article}

\usepackage[utf8x]{inputenc}
\usepackage[T1]{fontenc}
\usepackage[francais]{babel}
\usepackage{amssymb,amsmath}
\usepackage{fullpage}
\usepackage[final]{pdfpages}

\title{Calssification}
\date{01.10.2013}
\author{
    Florian Reinhard\\
    florian.reinhard@epfl.ch
}

\begin{document}
    \maketitle

    \section{Systèmes réluctants (sans aimant)}
    \begin{itemize}
        \item structure ferromagnétique
        \item 1 ou plusieurs bobinages
        \item la force vient du fait que le système tend à minimiser l'énergie
            électromagnétique
    \end{itemize}

    \begin{equation}
        F = \frac{1}{2}\sum^n_{i=1}\sum^n_{j=1}\frac{d\Lambda_{ij}}{dx}\Theta_i\Theta_j
        \label{force}
    \end{equation}
    
    Ça donne $ F = \frac{1}{2}\frac{dL}{dx}i^2 $ pour une bobine + du fer.

    \begin{itemize}
        \item Avantages:
            \begin{itemize}
                \item structure simple / bon marché
                \item inertie faible -> assez bonne dynamique
            \end{itemize}
        \item Inconvénients:
            \begin{itemize}
                \item force proportionnelle à $ i^2 $
                \item bruit et chocs sur $ i $
                \item rendement faible
                \item force toujours dans le même sens ($ F \propto i^2 $)
            \end{itemize}
    \end{itemize}

    \section{Électrodynamique (aimant fixe, bobine mobile)}
    On peut utiliser \emph{Laplace} pour résoudre tels systèmes.
    (Voir exemple du 2ème cours.)

    \begin{itemize}
        \item Avantages:
            \begin{itemize}
                \item $ F \propto i $
                \item inertie faible -> assez bonne dynamique
            \end{itemize}
        \item Inconvénients:
            \begin{itemize}
                \item Une bobine mobile est difficile à alimenter.
                \item guidage de la bobine
            \end{itemize}
    \end{itemize}

    \section{Électrodynamique (bobine fixe, aimant mobile)}

    \begin{equation}
        M = \frac{1}{2} \frac{d\Lambda_{a}}{d\alpha}\Theta_a^2
            + \frac{d\Lambda_{ab}}{d\alpha}\Theta_a \Theta_b
        \label{moment_bobine_fixe}
    \end{equation}

    La 1ère terme signifie la force due à l'aimant seul, qui veut se positionner
    et la 2ème corresponde à \emph{Laplace}.

    \begin{itemize}
        \item Avantages:
            \begin{itemize}
                \item force volumique la plus élevée
                \item pas d'alimentation mobile
                \item $ F \propto i $
            \end{itemize}
        \item Inconvénients:
            \begin{itemize}
                \item guidage de l'aimant
            \end{itemize}
    \end{itemize}

    \section{Systèmes hybrides ou réluctant polarisés}

    On a des aimants et bobines fixes et une pièce ferromagnétique mobile.
    (Par exemple un moteur pas-à-pas.)

    \begin{equation}
        M = \frac{1}{2}\frac{d\Lambda_{b}}{d\alpha}\Theta_b^2
            + \frac{1}{2}\frac{d\Lambda_{a}}{d\alpha}\Theta_a^2
            + \frac{d\Lambda_{ab}}{d\alpha}\Theta_a \Theta_b
        \label{moment_bobine_fixe}
    \end{equation}

    \begin{itemize}
        \item Avantages:
            \begin{itemize}
                \item bon rendement
                \item conversion information à position simple
            \end{itemize}
        \item Inconvénients:
            \begin{itemize}
                \item cher et compliqué
            \end{itemize}
    \end{itemize}
\end{document}
